\section{Assembler und C}
\subsection{Prinzipien des Datenaustausch}

\subsubsection{Datenaustausch via Register}
\begin{itemize}
  \item Aufrufer legt Parameter in die gemeinsamen Register
  \item Subroutine holt sich die Parameter von den Register
  \item Resultate werden wider in die Register abgelegt
\end{itemize}
\begin{tabular}{p{0.35\textwidth}|l}
	\textbf{Vorteile} 	& \textbf{Nachteile} \\
	\hline
	schnell				& Anzahl Parameter limitiert \\
	kleiner Overhead	& Parameter und Register müssen klar definiert werden \\
						& nicht ablaufinvariant \\ 
						& parallele Interupts sind gefährlich
\end{tabular}


\subsubsection{Datenaustausch via Register-Windowing}
\begin{itemize}
  \item Aufrufer legt Parameter in sein output register-window
  \item Subroutine holt sich die Parameter von seinem input register-window
  \item Resultate werden in das input register-window abgelegt
\end{itemize}
\begin{tabular}{p{0.35\textwidth}|l}
	\textbf{Vorteile} 	& \textbf{Nachteile} \\
	\hline
	schnell				& Anzahl Parameter limitiert auf die grösse des register window \\
	kleiner Overhead	& Parameter und Register müssen klar definiert werden \\
						& Level der Subroutine Aufrufe ist beschränkt (bei Rekursion) \\
						& eventuell nicht ablaufinvariant \\ 
						& parallele Interupts sind gefährlich \\
						& Register-Windowing selten unterstützt
\end{tabular}


\subsubsection{Datenaustausch via globalen Speicher}
\begin{itemize}
  \item Aufrufer legt Parameter in den globalen Speicher
  \item Subroutine holt Parameter vom globalen Speicher
  \item Resultate werden in den globalen Speicher gelegt
\end{itemize}
\begin{tabular}{p{0.35\textwidth}|l}
	\textbf{Vorteile} 	& \textbf{Nachteile} \\
	\hline
	virtuell unlimitierte Anzahl Parameter	& Parameterstruktur muss klar definiert werden \\
											& Overhead durch schreiben/lesen der Parameter \\
											& globale Variablen sind gefährlich \\
											& nicht ablaufinvariant
\end{tabular}


\subsubsection{Datenaustausch via Stack}
\begin{itemize}
  \item Aufrufer und Subroutine legen/holen sich die Parameter vom Stack
  \item Aufrufer und Subroutine müssen die Reihenfolge und das Format der Parameter einhalten
  \item \textbf{Achtung:} Alle Parameter müssen wieder vom Stack entfernt werden!
\end{itemize}
\begin{tabular}{p{0.35\textwidth}|l}
	\textbf{Vorteile} 	& \textbf{Nachteile} \\
	\hline
	virtuell unlimitierte Anzahl Parameter	& eventuell schwieriger zum verstehen \\
	ablaufinvariant, Rekursion möglich 		& \\
	klar und strukturiert					&
\end{tabular}

\subsection{Lokale Variablen und Stack Frames}
Eine Subroutine braucht eventuell einen bestimmten Speicher um lokale Variablen zu speichern. Für diesen Speicher gibt es zwei Möglichkeiten: \\
\textbf{Statische Allokation}
\begin{itemize}
  \item Fixer Adressraum für die lokalen Variablen
  \item Funktioniert nur wenn die Subroutine nicht ablaufinvariant und nicht rekursiv ist
\end{itemize}
\textbf{Dynamische Allokation}
\begin{itemize}
  \item Variablen auf den Stack ablegen
  \item Mit den Instruktionen \verb+LINK+ und \verb+UNLK+ kann ein Stack Frame, eine temporäre Region auf dem Stack angelegt werden
  \item Das Stack Frame wird für jeden Aufruf der Subroutine neu angelegt, somit ist Rekursion möglich
  \item Das Stack Frame ist über einen \textbf{Frame Pointer} erreichbar,
  welcher auf den "`Boden"' des Frames zeigt
  \item Static Variabeln sind IMMER ausserhalb des frames! 
\end{itemize}


\subsubsection{LINK und UNLK Instruktion}
\begin{tabular}{l|l}
	\verb+LINK+											& \verb+UNLK+ \\
	Stack Frame erstellen								& Stack Frame löschen \\
	\hline
	\verb+LINK A6,#-16+									& \verb+UNLK A6+ \\
	Stack Pointer um 4 verkleinern						& Inhalt von A6 in den Stack Pointer laden \\
	Inhalt der Adresse von Register A6 auf den Stack	& Den alten Wert von A6 in A6 abspeichern \\
	Stack Pointer in A6 ablegen							& Stack Pointer um 4 erhöhen \\
	Stack Pointer um 16 Adressen erhöhen				& 
\end{tabular}