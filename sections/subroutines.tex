\section{Subroutines}
Subroutinen stellen das gleiche dar wie Funktionsaufrufe in höheren Sprachen. Somit kann der gleiche Code mehrmals wiederverwendet werden.
\begin{itemize}
  \item Subroutine Aufruf: Inhalt von PC in den Stack, Sprung (jsr/bsr) in die Subroutine
  \item Subroutine Rückkehr: Inhalt vom Stack zurück in den PC.
\end{itemize}
\textbf{Achtung:} Eine Subroutine sollte nicht versuchen den "`Return-Pointer"' zu verändern. Dies führt zu Problemen wenn die Subroutine anders aufgerufen wird.

Normalerweise wird das Condition Code Register (CCR) durch die Subroutine beeinflusst. Möchte man das nicht kann das CCR auf dem Stack abgelegt werden und am Schluss wieder hergestellt werden: \\
\begin{lstlisting}
	  BSR GET_DATA
	GET_DATA: 
	  MOVE.W CCR,-(A7)  *CCR auf den Stack legen
	  RTR CCR           *zurueckschreiben und Subroutine verlassen
\end{lstlisting}

