\section{Exceptions}
\textbf{Exceptions} und \textbf{Interrupts} sind sehr ähnlich: Beide verändern den normalen Programmablauf.
Eine \textbf{Exception} ist ein Aufruf auf das Betriebssystem und ist einer Subroutine sehr ähnlich. 
\textbf{Interrupts} sind Hardware-Exceptions welche von einem externen Gerät ausgelöst werden.
Jede \textbf{Exception} hat einen \textbf{Exception Handler} welcher richtig auf die Exception reagiert.
\subsection{Wozu?}
Um HW/SW-Ereignisse an die SW zu signalisieren und innerhalb des normalen
Programmablaufs zu verarbeiten.\\
Zum Aufruf von Subroutinen ohne Kenntnis der Startadresse\\
Interne/Externe Fehler mittels SW zu behandeln.
\subsection{Prioritäten}
Die Prioritäten der Exceptions und Interrupts sieht man in ``Exception Grouping and Priority'' in
der Prog. Rev. Card. Bei den Autovectored Interrupts schaut man in der Tabelle ``Interrupt
Encoding''. Ausserdem gelten die folgenden Regeln:
\begin{itemize}
	\item $IRQ_i$ Je grösser i desto höher die Priorität
	\item Eine ISR tieferer Priorität kann nur von einem IRQ höherer Priorität unterbrochen werden
	\item Eine ISR kann nicht von einem IRQ gleicher oder tieferer Priorität unterbrochen werden
\end{itemize}
\subsection{Aufruf}
Eine Exceptions darf nicht wie eine normale Funktion aufgerufen werden, da sie einen anderen
Rücksprungbefehl (RTE anstelle RTS) verwendet, welcher auch das Statusregister vom Stack holt.
An der Vektoradresse liegt die Adresse der aufzurufenden Subroutine.\\
Vektoradresse = 4xVektornummer = Vektortabelle (0x000000 bis 0x0003FF) +
Startadresse-Offset
\subsection{Exception Routine}
\begin{itemize}
  \item Hat keine explizite Adresse, sondern vector
  \item speichern diverse Register to supervisor stack (siehe Karte)
  \item s-Bit = 1
  \item Rücksprung mit RTE nicht mit RTS
\end{itemize}

\subsection{Deklaration in C}
\begin{lstlisting}[language=C]
#define exc_vector(x) (*(void (**)(void))((x) * 4)) /* Adresse aus Index berechnen */
#define DIV0_VECNO 5                                /* Index von der Exception Vector Assignment */

#pragma interrupt
void exception_routine(void)
{
...
}
#pragma endinterrupt

exc_vector(DIV0_VECNO)=exception_routine; /* Vektortabelleneintrag setzen exc_div0 ist ein funktionsname*/
\end{lstlisting}

\subsection{Beispiele} 

\subsubsection{Exception Routine für Software-Interrupt trap \#3}

\begin{lstlisting}[language=C]
#define TRAP3_VECNO (32+3)

int error;                                          /*globale Variable fier Zugriff aus Exception Routine*/
typedef void (*t_funcptr)();                        /* "t_funcptr" ist ein Zeiger auf eine Funktion */

t_funcptr *p_vector_trap3 = (t_funcptr *)(4 * TRAP3_VECNO);

#pragma interrupt

void exc_trap3 (void)
{
	ANZEIGE=(unsigned char)error;
}
#pragma endinterrupt

/* Im Hauptprogramm: Vektor in Tabelle eintragen */
	*p_vector_trap3 = exc_trap3;

/* Aufruf z.B. mit folgender Sequenz: */
	error = 0x18;
	asm("trap #3");
\end{lstlisting}


\subsubsection{Exception Routine für Vektor-Nr.8}
Die Routine legt die unerlaubte Instruktion ( Privilege Violation) auf Adress
0x0040 0804 ab, und springt so zurück, dass die unerlaubte Instruktion nicht mer
ausgeführt wird (unerlaubte Instruktion 2Byte lang)
\begin{lstlisting}
ill_ex: 
        move.l  a0,sp@-       *Register a0 auf Stack sichern
        move.l  sp@(6),a0     *Statt movea kann man immer move schreiben
                              *auf dem Stack deponiert werden (Offset dann +6)
        move.w  a0@,0x400804  *unerlaubte Instruktion sichern
        move.l  sp@+,a0       *Gesichertes a0 zurueckholen vom Stack
        addq.l  #2,sp@(2)     *unerlaubte Instruktion nicht mehr ausfuehren
        rte                   *Return from Exception
        
*Initialisierung der ill_ex: Vektornummer 8
        move.l  #ill_ex,4*8
        
\end{lstlisting}
\begin{itemize}
  \item bus error, adress error, division durch Null, Rechteverletzung
  \item soft interrupts z.B. für Systemcalls
\end{itemize}

