\section{Exceptions}
\textbf{Exceptions} und \textbf{Interrupts} sind sehr ähnlich: Beide verändern den normalen Programmablauf.
Eine \textbf{Exception} ist ein Aufruf auf das Betriebssystem und ist einer Subroutine sehr ähnlich. 
\textbf{Interrupts} sind Hardware-Exceptions welche von einem externen Gerät ausgelöst werden.
Jede \textbf{Exception} hat einen \textbf{Exception Handler} welcher richtig auf die Exception reagiert.
\subsection{Prioritäten}
\begin{itemize}
	\item $IRQ_i$ Je grösser i desto höher die Priorität
	\item Eine ISR tieferer Priorität kann nur von einem IRQ höherer Priorität unterbrochen werden
	\item Eine ISR kann nicht von einem IRQ gleicher oder tieferer Priorität unterbrochen werden
\end{itemize}
\subsection{Aufruf}
Eine Exceptions darf nicht wie eine normale Funktion aufgerufen werden, da sie einen anderen Rücksprungbefehl (RTE anstelle RTS) verwendet, welcher auch das Statusregister vom Stack holt.

\subsection{Deklaration in C}

\begin{lstlisting}[language=C]
#define exc_vector(x) (*(void (**)(void))((x) * 4)) /* Adresse aus Index berechnen */
#define DIV0_VECNO 5                                /* Index von der Exception Vector Assignment */

#pragma interrupt
void exception_routine(void)
{
...
}
#pragma endinterrupt

exc_vector(DIV0_VECNO)=exception_routine; /* Vektortabelleneintrag setzen exc_div0 ist ein funktionsname*/
\end{lstlisting}

\subsection{Beispiele} 

\subsubsection{Exception Routine für Software-Interrupt trap \#3}

\begin{lstlisting}[language=C]
#define TRAP3_VECNO (32+3)

int error;                                          /*globale Variable fier Zugriff aus Exception Routine*/
typedef void (*t_funcptr)();                        /* "t_funcptr" ist ein Zeiger auf eine Funktion */

t_funcptr *p_vector_trap3 = (t_funcptr *)(4 * TRAP3_VECNO);

#pragma interrupt

void exc_trap3 (void)
{
	ANZEIGE=(unsigned char)error;
}
#pragma endinterrupt

/* Im Hauptprogramm: Vektor in Tabelle eintragen */
	*p_vector_trap3 = exc_trap3;

/* Aufruf z.B. mit folgender Sequenz: */
	error = 0x18;
	asm("trap #3");
\end{lstlisting}

\begin{itemize}
  \item bus error, adress error, division durch Null, Rechteverletzung
  \item soft interrupts z.B. für Systemcalls
\end{itemize}

