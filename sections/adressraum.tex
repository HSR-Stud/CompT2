\section{Adressraum}
\begin{multicols}{2}
\begin{itemize}
  \item Bussystem: Verbindet die Komponenten
  	\begin{itemize}
  		\item CPU: Control Unit, Processing Unit
  			\begin{itemize}
  				\item Init Phase: CPU-Startup
  				\item Infinite Loop: "`fetch-execute"'-Cycle
			\end{itemize}
  		\item Memory: Programm, Daten
  		\item I/O: Inptut/Output
	\end{itemize}
  \item Komponenten eines Prozessor-Bussystem:
  	\begin{itemize}
  		\item Adress Bus: Uni-Direktional, Bestimmt maximal adressierbarer Speicher
  		\item Data Bus: Bi-Direktional, Überträgt Daten zwischen CPU und Memory
  		\item Control Bus: Überträgt Timing- und andere Steuer-Signale von der CPU an die I/O-Geräte
	\end{itemize}
\end{itemize}
\end{multicols}

\subsection{Stack}
Der Stack wächst immer von der höchsten Adresse zur tiefsten Adresse hin, der Stackpointer zeigt immer auf den obersten Eintrag im Stack (Top of Stack \textbf{ToS}).
Der Stack ist ein typischer \textbf{last-in first-out (LIFO)} Speicher. \\
\begin{tabular}{lll|l}
	\textbf{Push}	& move.w	& source, -(SP)			& siehe auch \textbf{movem}: push/pull list of registers \\
	\textbf{Pull}	& move.w	& (SP)+, destination	& movem.l \quad a1-a4/a6/d0-d3, -(a7)
\end{tabular}

